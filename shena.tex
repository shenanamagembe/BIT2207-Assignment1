\documentclass[options]{article}
\begin{document}
\textbf{A Research Report about the use of mobile phones during lectures and/or exams at the         university submitted for qualification in the course unit BIT 2207 Research Methodology}
\section{\textbf{Executive Summary 
 The aim of this report was to investigate Makerere students attitudes to personal mobile phone use in class rooms.  A student survey on attitudes towards the use of mobile phones in the class was conducted. The results indicate that the majority of students find mobile phone use a major issue in class.  The report concludes that personal mobile phones are disruptive and should be turned off in lectures.  It is recommended that Makerere develops a rule banning the use of mobile phones except in exceptional circumstances.
}}
\section{\textbf{Introduction  
There has been a massive increase in the use of mobile phones over the past years and there is every indication that this will continue. According to research almost 100 of college students carry mobile phones. Currently at Makerere 89 of students have mobile phones.  Recently a number of lecturers have complained about the use of mobile phones in class and asked what the official university rule is. At present there is no official university rule regarding phone use.  This report examines the issue of mobile phone usage in lecturers. It does not seek to examine the use of mobile phones at the university at other times. 
}}

\section{\textbf {Methodology
This research was conducted by questionnaire and investigated student attitudes to the use of mobile phones in lectures.  A total of 10 questionnaires were distributed in Bsc.Computer Science.  The questionnaire provided open ended responses for additional comments.  No personal information was collected; the survey was voluntary and anonymous.
l}}.
\section{\textbf{Results  
There was an 65 response rate to the questionnaire. My dedaction from the results is that mobile phones are considered to be disruptive and should be turned off in lectures. The survey also allowed participants to identify any circumstances where mobile phones should be allowed in lectures and also assessed students attitudes towards receiving personal phone calls in class in open ended questions.  These results showed that students thought that in some circumstances, eg medical or emergencies, receiving personal phone calls was acceptable, but generally receiving personal phone calls was not necessary.
}}
\section{\textbf { Discussion   
It can be seen from the results that personal mobile phone use is considered to be a problem; however it was acknowledged that in some situations it should be permissible.  60 of recipients considered mobile phones to be highly disruptive and there was strong support for phones being turned off in lectures.  Only 30 thought that mobile phone usage in lectures was not a problem, whereas 45 felt it was an issue.  The results are consistent throughout the survey.  These findings are consistent with other studies in that claims that 29 of class time is wasted through unnecessary mobile phone interruptions. This affects time management, productivity and student focus. 
l}}.
\section{\textbf{ Conclusion 
The use of mobile phones in lecturers is clearly disruptive and they should be switched off.  Most students felt it is not necessary to receive personal phone calls in lectures except under certain circumstances, but permission should first be sought from the lecturers. 
}}
\section{\textbf{Recommendations  
It is recommended that the university develops an official policy regarding the use of mobile phones in lectures. The policy should recommend:  • mobile phones are banned in lectures  • mobiles phone may be used in exceptional circumstances but only with the permission of lecturers  Finally, the policy needs to apply to all students in the university.
}}
















\end{document}